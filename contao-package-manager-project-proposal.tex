\documentclass[
paper=a4,
draft=false,%     Entwurfsmodus
fontsize=10pt% Größe der Grundschrift
]{scrartcl}

\def\logo{logos/colorlogo_rgb}% Logo bei Normaldruck.
%\def\logo{logos/colorlogo_cmyk}% Logo bei Profidruck.

\usepackage{cca-page} % design nach vorschrift
\usepackage{multirow}

\newcommand{\contaoPackageManagerProject}{\textbf{\texttt{Contao Paket Manager Projekt}}}
\newcommand{\contaoPackageManager}{\textbf{\texttt{Contao Paket Manager}}}
\newcommand{\packageManager}{\textbf{\texttt{Paket Manager}}}
\newcommand{\moduleInstaller}{\textbf{\texttt{Module Installer}}}
\newcommand{\store}{\textbf{\texttt{Extension Store}}}
\newcommand{\jsonGenerator}{\textbf{\texttt{Composer.json Generator}}}
\newcommand{\documentation}{\textbf{\texttt{Dokumentation}}}
\newcommand{\databaseUpdateTool}{\textbf{\texttt{Database Update Tool}}}
\newcommand{\migrationToolkit}{\textbf{\texttt{Migration Toolkit}}}
\newcommand{\composerClient}{\textbf{\texttt{Composer Client}}}
\newcommand{\composerPlugin}{\textbf{\texttt{Composer Plugin}}}

\begin{document}

\title{%
An die:\\
Contao Association\\
Sonnhalderain 15\\
CH - 3250 Lyss\\
\href{http://c-c-a.org}{\includegraphics[width=.6\textwidth]{\logo}}\\
Antrag auf Förderung des \\
\textit{Contao Paket Manager Projekts} \\
durch die Contao Association\\%
}
\date{24 Oktober 2014}
\author{bit3 UG, CyberSpectrum}
\maketitle

\pagebreak

% --------------------------------------------------------------------------------
%
%     Präambel
%
% --------------------------------------------------------------------------------

\section*{Präambel}
\label{sec:preamble}

Gefördert werden soll das \contaoPackageManagerProject{}, das durch die CCA betreut wird.

\begin{emquotation}
Ursprünglich hieß dieses Projekt \texttt{Contao Composer Projekt} und bestand aus den Projekten \composerClient{} (\texttt{contao-community-alliance/composer-client}) und dem \composerPlugin{} (\texttt{contao-community-alliance/composer-plugin}). Wegen der Benennung kam es oft zu Verwechselung mit dem \texttt{PHP Composer Projekt}\footnote{\hreflink{https://getcomposer.org}} was zu viel Verwirrung geführt hat. In einer gemeinsamen Diskussion hat man beschlossen das Projekt um zu benennen\footnotemark um hier eine klare Trennung zu schaffen.
\end{emquotation}

\footnotetext{\hreflink{https://github.com/contao-community-alliance/composer-client/issues/235}}

Das Projekt besteht aus mehreren Teilprojekten:

\begin{itemize}
\item ...dem \packageManager{}, \\
dabei handelt es sich um eine eigenständige Applikation - vergleichbar mit dem Install-Tool und Live-Update-Tool - um Contao, seine Pakete und Erweiterungen zu verwalten. \\
Den meisten Nutzern dürfte der Vorgänger des \packageManager{}, der \composerClient{} als Menüpunkt \uquot{Paketverwaltung} bekannt sein.

\item ...dem \moduleInstaller{}, \\
dieser stellt in Composer die Funktion zur Verfügung, Contao Erweiterungen über Composer zu installieren. \\
$ \ast $ Erweiterungen die exklusiv für Contao 4 oder neuer entwickelt werden benötigen den \moduleInstaller{} nicht mehr.

\item ...dem \store{}, \\
dies ist ein an das Contao Ökosystem angepasster Paketserver auf Basis von Packagist.

\item ...der \documentation{}.
\end{itemize}

\pagebreak

\subsection*{Welches Ziel hat dieser Antrag?}

Bereits Anfang 2014\footnote{\hreflink{https://contao.org/de/news/zwischenmeldung-zum-extension-repository-3.html}} haben wir in einem klärenden Gespräch festgestellt, dass der vorhandene \composerClient{} zwar funktioniert, aber es vor allem am Ökosystem hapert. Das vorhandene Ökosystem rund um das ER2 ist deutlich umfangreicher als das, was wir um den \composerClient{} bisher aufbauen konnten.

Dieser Antrag befasst sich - soweit es aktuell planbar ist - damit, das Ökosystem rund um den \contaoPackageManager{} aufzubauen, welcher aus dem \composerClient{} hervor geht. So dass wir in absehbarer Zeit - 2015/2016 - das alte ER2 auch tatsächlich ablösen können.

Vor allem die \uquot{mangelnde} Usability, die in den Kommentaren des Newsbeitrags vom 20. März 2014 so oft angesprochen wurde möchten wir \uquot{aus dem Weg schaffen}.

\pagebreak
\tableofcontents
\pagebreak

% --------------------------------------------------------------------------------
%
%     tl;dr - Zusammenfassung
%
% --------------------------------------------------------------------------------

\section{Zusammenfassung}
\label{sec:summary}

\textit{Der Antrag ist dir zu lang? Hier findest du eine Zusammenfassung!}

Wie im Präambel beschrieben umfasst der Antrag das gesamte Contao Composer Projekt. Nicht nur aktuelle, auch kurzfristige, sowie mittelfristige und langfristige Ziele.

Der Antrag verfolgt dabei 2 Ziele:
Erstens wollen wir die Kritiken aufgreifen und die Bedienbarkeit verbessern. Dies soll zum einen durch Verbesserung der Oberfläche, zum anderen durch bereitstellen einer Dokumentation geschehen.
Zweitens wollen wir das Ökosystem aufbauen, dass das aktuelle ER2 zur Zeit bietet, so dass wir im Zeitrahmen 2015/2016 ernsthaft darüber reden können, das alte ER2 abzulösen. An dieser Stelle erst mal Entwarnung: Das ER2 wird uns noch einige Jahre erhalten bleiben um älteren Contao Installationen nicht \uquot{den Stecker zu ziehen}.

Composer ist nicht ganz unumstritten, da es teilweise sehr ressourcenhungrig ist. Allerdings stehen uns kaum Alternativen zur Verfügung. Native Paketverwaltungen wie PEAR/PECL fallen auf Shared Hostings weg. Da Composer rein in PHP geschrieben ist, lässt es sich auf nahezu allen PHP Plattformen betreiben. Man muss auch sagen dass sich Composer in den vergangenen 2 Jahren sehr gut weiterentwickelt hat und deutlich verbessert wurde. Die Systemanforderungen sind gesunken und die Performance gestiegen. Wir setzen hier auf die Erfahrung und Arbeit von etwa 10 Entwicklern die regelmäßig zu Composer beitragen. Und über 100 weiteren Entwicklern die hier und da was zum Composer beitragen\footnote{\hreflink{https://github.com/composer/composer/graphs/contributors}}.

Wir sind nicht die einzigen die auf den \uquot{Composer Zug} aufspringen. Drupal, Typo3 NEOS, Laravel, Yii, Doctrine, Symfony und zahllose weitere Projekte nutzen Composer. Es gibt bereits über 40.000 Pakete auf packagist.org zu finden! Es ist anzunehmen, dass durch die große Verbreitung auch die Shared Hoster zukünftig dazu gezwungen sein werden, Composer besser zu unterstützen.

Der Umfang des Antrags würde das Jahresbudget der Association deutlich sprengen. Deshalb wird dieser Antrag in \uquot{Entwicklungspakete} aufgesplittet, die dann zur Förderung eingereicht werden oder durch andere Unternehmen gefördert werden können.
Für die Community ist die Gesamtheit des Antrags aber immer noch ein Leitfaden, was alles noch folgen wird.
Dabei sollte jedoch beachtet werden, dass der Antrag nicht starr ist. Einige mittel- und langfristige Positionen sind noch nicht vollständig ausgearbeitet und geplant und das ist auch beabsichtigt. Die weiterführenden Positionen wollen wir in Zusammenarbeit mit der Community ausarbeiten und planen.

Dabei helfen uns die \uquot{Entwicklungspakete} die zeitgleich zu einem agilen Entwicklungsmodel führen. Nach jedem Entwicklungsschritt können wir ein Resümee ziehen und gemeinsam die nächsten Schritte planen.

\newpage

% --------------------------------------------------------------------------------
%
%     Composer, was ist das?
%
% --------------------------------------------------------------------------------

\section{Composer, was ist das?}
\label{sec:why-composer}

\subsection{Verwirrende Nomenklatur - Was ist Composer?}

\uquot{Composer} hört man als Begriff in der Contao Community in letzter Zeit sehr häufig. Dabei stammt Composer gar nicht aus dem Contao Umfeld. Composer\footnote{\hreflink{https://getcomposer.org}} ist eine sogenannte Paketverwaltung speziell für PHP - nicht nur speziell für Contao. Das Composer Projekt wurde Anfang 2011 von Nils Adermann und Jordi Boggiano initiiert und erfreut sich seit 2012 wachsender Beliebtheit und Verbreitung. Abgesehen von PEAR\footnote{\hreflink{http://pear.php.net/}}, welches nativ auf dem System ausgeführt wird, gab es für PHP über die letzten Jahre hinweg keine vernünftige Paketverwaltung. Composer ist vollständig in PHP geschrieben und lässt sich daher im Gegensatz zu PEAR auch auf restriktiven Systemen (Shared Hoster) verwenden. Zum Composer-Ökosystem gehört unter anderem Packagist\footnote{\hreflink{https://packagist.org/}}, das als primäres Repository für Pakete dient.

\begin{info}
In diesem Dokument ist mit \textbf{Composer} immer das Composer Projekt von Nils und Jordi gemeint.
In der Contao Community ist damit umgangssprachlich der \textbf{Contao Composer Client} gemeint.
Leider führt das des öfteren immer wieder zu Verwirrung. \\
\\
Unter dem Begriff \textbf{Contao Composer Projekt} - umgangssprachlich auch als \textbf{Composer Projekt} abgekürzt - vereinen wir alle zugehörigen Teilprojekte (siehe \nameref{sec:preamble}).
\end{info}

\subsection{Was ist aus dem ER3 geworden?}

Das ER3, auch bekannt als RFCCC-1\footnote{\hreflink{https://c-c-a.org/rfccc1}} war ein Vorstoß aus 2011 um den Weg für einen neuen Erweiterungskatalog zu ebnen, im gleichen Jahr hat Composer das Licht der Welt erblickt. Im Jahr 2012 hatte Composer dann seinen großen Aufschwung und wir hatten neugierige Blicke auf dieses aufstrebende Projekt geworfen. Anfang 2013 entstand dann die erste Integration von Composer für Contao.

\begin{emquote}{Christian Schiffler, Tristan Lins}
Als wir den RFCCC-1 (ER3 Dokumentation) schrieben, haben wir festgestellt dass die ganze Thematik \uquot{Paketverwaltung und Abhängigkeitsmanagement} technisch deutlich komplexer ist, als es auf den ersten Blick den Eindruck macht. Nils und Jordi haben hier eine großartige Vorarbeit geleistet, viele Stunden investiert und obendrein eine große Anzahl an Mitstreitern um sich geschart die regelmäßig dazu beitragen, Composer zu verbessern - wir gehören auch dazu. Composer mag durchaus auch seine Schwächen haben - bspw. relativ hohe Systemanforderungen - aber wir könnten es auch nicht besser machen, besonders nicht bei gleicher Funktionalität.
\end{emquote}

Für uns war, ist und bleibt Composer aktuell die beste Grundlage die wir nutzen können. Das ER3 ist also nicht gestorben, viel mehr haben wir mit Composer die Grundlage zur Hand, um das ER3 in erschwinglicher Zeit und akzeptablen finanziellen Aufwand darauf aufzubauen.

Schon heute wird der Contao Composer Client in immer mehr Installationen eingesetzt (>15.000 Downloads\footnote{\hreflink{https://c-c-a.org/composer-counter}}). Einige Erweiterungen die bereits über Composer vertrieben werden, werden im ER2 nur noch sporadisch aktualisiert. Andere Erweiterungen sind gar nicht mehr im ER2 verfügbar. Die bekanntesten Vertreter sind vermutlich Avisota und MetaModels, aber auch einige andere Erweiterungen die nicht aus der Hand der Antragsteller stammen, sind nur noch über Composer verfügbar! Bspw. die Contao Bootstrap Erweiterung\footnote{\hreflink{https://packagist.org/packages/contao-bootstrap/}} von David Molineus.

\newpage

% --------------------------------------------------------------------------------
%
%     Vorteile für Nutzer
%
% --------------------------------------------------------------------------------

\section{Vorteile für Nutzer}
\label{sec:pros-for-users}

\begin{minipage}{.02\linewidth}
  \rotatebox{90}{\textbf{Vorteile auf einen Blick}}
\end{minipage}
\begin{minipage}{.98\linewidth}
  \begin{dinglist}{'64}
  \item Abhängigkeitskonflikte zwischen Erweiterungen werden automatisch aufgelöst und nur kompatible Versionen werden installiert - Stichwort: Versionshopping\footnotemark.
  \item Man erhält schneller Updates und Bugfixes, im Idealfall sofort nach dem ein Bug behoben wurde - Stichwort: Semantische Versionierung\footnotemark.
  \item Im \textit{Standalone Modus} kann man eine defekte Installation auch ohne FTP und händische Korrektur wiederherstellen.
  \item Zukunftssicher dank Open Source und einer breiten Entwicklergemeinschaft - 10 aktive und über 100 weitere Entwickler.
  \end{dinglist}
\end{minipage}

\addtocounter{footnote}{-1}
\footnotetext{Unter Versionshopping verstehen wir die Situation, in der das ER2 dauerhaft eine Erweiterung erst upgraden, danach direkt wieder downgraden will und umgekehrt. Versionshopping entsteht, wenn eine Erweiterung von zwei anderen Erweiterungen als Abhängigkeit bezogen wird und dazu auch noch in 2 unterschiedlichen Versionen. Das ER2 springt dann zwischen den 2 angeforderten Versionen der abhängigen Erweiterung stendig hin und her.\vspace*{3mm}\label{fn:versionshopping}}
\stepcounter{footnote}
\footnotetext{\href{http://semver.org/}{SemVer2 (semver.org)} beschreibt eine Art wie man Versionen einer Software festlegt. Der 3. Level einer Version (also der Z Level: x.y.\textbf{Z}) dient dabei zum versionieren von Bugfixes. Im Idealfall hat jeder Bugfix seine eigene Versionsnummer. D.h. verfolgt ein Entwickler dieses Prinzip und hat er einen Bug behoben, kann er ihn auch gleich in einer neuen Version veröffentlichen. Nutzer müssen nicht mehr auf das nächste \uquot{große Release} warten oder sich mühevoll den Bugfix händisch installieren.\vspace*{3mm}\label{fn:semver}}

\textit{Ein schlanker Core und hochwertige/umfangreiche Erweiterungen}, dafür steht Contao seit 2006. Um die Bereitstellung des Erweiterungskatalogs kümmert sich seit geraumer Zeit das so genannte \uquot{Extension Repository 2} - kurz \uquot{ER2}. Das ER2 stellt aktuell die Erweiterungsliste auf Contao\footnote{\hreflink{https://contao.org/de/extension-list.html}}, so wie den Erweiterungskatalog und die Erweiterungsverwaltung im Contao Backend. Doch das ER2 hat seine Schwächen, so werden Abhängigkeiten nicht richtig aufgelöst - Stichwort: Versionshopping\footref{fn:versionshopping} - oder es lassen sich mit der Contao Version oder mit anderen Erweiterungen inkompatible Erweiterungen installieren.

Unterm Strich muss man sagen, dass die Vorteile von Composer primär die Entwickler betreffen. Doch was für den Entwickler ein Vorteil ist, ist letztlich auch für den Nutzer ein Vorteil. Aber auch die Schwächen des aktuellen CCC für die Nutzer sind uns bewusst und gerade diese wollen wir mit Hilfe dieses Antrags beheben. Dazu zählen unter anderem eine deutliche Verbesserung der Übersichtlichkeit beim Finden, Installieren und Verwalten von Erweiterungen. Zum anderen wollen wir die Kompatibilität zu den zahlreichen Shared Hostern erhöhen.

Aber worum geht es konkret?

Durch die veraltete und unzureichende Implementierung der Abhängigkeitsverwaltung im ER2 ist es aktuell möglich, wenn nicht gar wahrscheinlich, dass sich Nutzer Erweiterungen die zu einander oder zur aktuellen Contao Version inkompatibel sind installieren und somit potentiell die Installation lahm legen - Stichwort: \uquot{gelbe Box}. Diese Probleme werden durch Composer weitestgehend vermieden, so lange die Entwickler selbst keinen Mist bauen. Denn Composer hat eine \uquot{strikte} und zugleich \uquot{offener} Auflösung der Abhängigkeiten. Richtig angewendet sorgt das dafür, dass man als Nutzer immer mit dem neuesten, stabilsten und zu meiner Installation kompatiblen Set an Erweiterungen arbeiten kann.

Sollte es dennoch zu Problemen kommen - den auch Composer kann uns nicht vor allen Problemen schützen - und sollte das Backend dann nicht mehr nutzbar sein, kann man mit Hilfe des \textit{Standalone Modus} die defekte Erweiterung auch wieder entfernen und das Backend \uquot{wiederbeleben}. Den \textit{Standalone Modus} kann man sich ähnlich wie das Install Tool vorstellen, nur halt für die Erweiterungsverwaltung. Mit dem ER2 ist dies aktuell nicht möglich und man muss von Hand via FTP nachhelfen. Gerade für nicht-Entwickler ist hier nicht selten Nachfragen im Forum, IRC, Twitter und Co. erforderlich.

Ein weiterer Punkt ist die Veröffentlichung von neuen Versionen. Für Entwickler ist dies aktuell relativ zeitaufwendig, abgesehen vom \uquot{taggen}\footnote{Damit bezeichnet man das Markieren einer Version zu einem gewissen Stand des Quellcodes.} der Version müssen Entwickler die neue Version auch noch auf der Contao Website eintragen, von Hand hochladen\footnote{Es gibt zwar mitlerweile einen github Import, der aber nicht in jeder Situation funktioniert.} und dann auch noch veröffentlichen. Composer und Packagist\footnote{\href{https://packagist.org}{Packagist.org} ist der offizielle Paketserver zu Composer.} automatisieren den Prozess nach dem \uquot{taggen} vollständig. Das macht es Entwicklern einfacher, gerade Bugfixes schnell und unkompliziert zu veröffentlichen - Stichwort: Semantische Versionierung\footref{fn:semver}.

Zu guter Letzt haben wir mit dem ER2 noch ein großes Problem, was den wenigsten bekannt ist. Das ER2 wurde ursprünglich exklusiv für Contao entwickelt und unterliegt einer proprietären Lizenz, die ausschließlich die Nutzung auf contao.org gestattet. Das bedeutet, es ist unmöglich das ER2 - oder zumindest den Server - weiter zu entwickeln, denn Leo Feyer darf den Quellcode des ER2 Servers nicht veröffentlichen. Mit Composer setzen wir auf Open Source Software, die wir an unsere Bedürfnisse anpassen können, wenn es notwendig wird.

\newpage

% --------------------------------------------------------------------------------
%
%     Vorteile für Entwickler
%
% --------------------------------------------------------------------------------

\section{Vorteile für Entwickler}
\label{sec:pros-for-developers}

\begin{minipage}{.02\linewidth}
  \rotatebox{90}{\textbf{Vorteile auf einen Blick}}
\end{minipage}
\begin{minipage}{.98\linewidth}
  \begin{dinglist}{'64}
  \item Abhängigkeitskonflikte zwischen Erweiterungen werden zuverlässig erkannt und vermieden.
  \item Große Auswahl an Contao Erweiterungen und PHP Bibliotheken die man als Abhängigkeit mit installieren lassen kann (mehr als 40.000) - Stichwort: Synergie.
  \item Die große Community sorgt für eine ständige Weiterentwicklung von Composer und dessen Ökosystems - Stichwort: Zukunftssicher.
  \item Einfache und \uquot{automatisierte} Veröffentlichung von neuen Versionen - Stichwort: Semantische Versionierung.
  \end{dinglist}
\end{minipage}

Den meisten Entwicklern sollten die Schwächen des ER2 bekannt sein. Die schlechte und unzureichende Auflösung von Abhängigkeiten zwischen Erweiterungen sorgt regelmäßig für Probleme. Und die aufwendige Pflege zur Veröffentlichung neuer Versionen ist nichts weiter als eine große ABM\footnote{Arbeitsbeschaffungsmaßnahme}.

Composer beherrscht das Auflösen von Abhängigkeiten perfekt. Mit Hilfe der \textit{Package links}\footnote{\hreflink{https://getcomposer.org/doc/04-schema.md\#package-links}} kann man als Entwickler die Abhängigkeiten genau zu definieren. Hinzu kommt dass man nicht nur die Abhängigkeiten auf andere Contao Erweiterungen definieren kann. Es ist sogar möglich mehr als 40.000 PHP Bibliotheken zu nutzen. Daraus ergeben sich für Entwickler ganz neue Möglichkeiten - Stichwort: Synergie.

Die große Community sorgt außerdem dafür, dass Composer und sein Ökosystem ständig weiterentwickelt werden. Bei Problemen hilft also nicht nur die Contao Community, sondern auch die Composer Community. Im \#composer\footnote{\hreflink{irc://irc.freenode.net/\#composer}} IRC Channel tummeln sich durchschnittlich 100 bis 150 Nutzer täglich, im \#composer-dev\footnote{\hreflink{irc://irc.freenode.net/\#composer-dev}} Channel sind es um die 30 bis 50 Entwickler täglich. Zum Vergleich bei Contao tummeln sich in \#contao\footnote{\hreflink{irc://irc.freenode.net/\#contao}} nur um die 30 bis 50 Nutzer und in \#contao.dev\footnote{\hreflink{irc://irc.freenode.net/\#contao.dev}} nur 15 bis 25 Entwickler. Die große, wachsende Composer Community liefert auch immer wieder neue \uquot{Best-Practice} Beispiele, Anleitungen und Tutorials.

Veröffentlichen von neuen Versionen wird mittels Composer so einfach wie nie zuvor, denn die Erweiterungen können dort belassen werden, wo sie heutzutage eh meist schon sind: Auf github, bitbucket oder einem anderen Anbieter. Einzige Bedingung dabei ist, dass der Quellcode öffentlich erreichbar ist. Im Gegensatz zum ER2, wo jede Version einzeln von Hand auf dem ER2 Server gepflegt werden muss, reicht bei Composer ein einmaliges Registrieren auf dem Paketserver (bspw. packagist.org). Neue Versionen werden ausschließlich durch setzen von Tags\footnote{Markierung eines exakten Standes der Entwicklung unter einem Namen, oftmals eine Versionsnummer in der Art von 1.0.0} im VCS System erzeugt. Da für gewöhnlich diese Tags von Entwicklern ohnehin gesetzt werden, fällt die ganze Verwaltung der Versionen auf dem Paketserver weg. Man kann also mit Recht behaupten: Es fällt Arbeit weg und kommt keine neue Arbeit hinzu, um die einzelnen Versionen zu verwalten!

\begin{info}
* Für Anbieter kommerzieller Erweiterungen gibt es die Möglichkeit - über private Satis Repositories und Artifact Pakete - den eigenen Code vor fremden Zugriffen zu schützen und trotzdem die Erweiterung über Composer zu verteilen. Diese Möglichkeiten werden im Rahmen der Dokumentation genau erklärt. Außerdem soll es langfristig einen \uquot{App Store} geben um kostenpflichtige Erweiterung über eine zentrale Plattform zu vermarkten und zu verteilen, siehe \nameref{sec:long-term-goals}.
\end{info}

\newpage

% --------------------------------------------------------------------------------
%
%     Antragsteller
%
% --------------------------------------------------------------------------------

\section{Antragsteller}
\label{sec:proposer}

\subsection*{Christian \uquot{xtra} Schiffler}

\begin{wrapfigure}{r}{.3\textwidth}
  \vspace{-50pt}
  \hfill
  \includegraphics[width=.2\textwidth]{bilder/cyberspectrum}
\end{wrapfigure}

Primärer Maintainer des CCP \\
Unternehmen: CyberSpectrum (\href{https://www.cyberspectrum.de}{www.cyberspectrum.de}) \\
E-Mail: \href{mailto:c.schiffler@cyberspectrum.de}{c.schiffler@cyberspectrum.de}

Softwareentwickler seit über 21 Jahren, bei Contao seit über 6 Jahren, Entwickler von MetaModels\footnote{\hreflink{https://now.metamodel.me}}, Betreuer des Contao Community Wiki\footnote{\hreflink{http://www.contaowiki.org/}}, Mitgründer der CCA\footnote{\hreflink{https://c-c-a.org/}\label{fn:cca}} und Mitglied der Contao Arbeitsgruppe \uquot{Core-Entwicklung}\footnote{\hreflink{https://contao.org/de/team.html\#workgroup-core}\label{fn:contao-workgroup-core}}.

\subsection*{Tristan \uquot{tril} Lins}

\begin{wrapfigure}{r}{.3\textwidth}
  \vspace{-40pt}
  \hfill
  \includegraphics[width=.2\textwidth]{bilder/bit3}
\end{wrapfigure}

Primärer Maintainer des CCC \\
Unternehmen: bit3 UG (\href{https://bit3.de}{bit3.de}) \\
E-Mail: \href{mailto:tristan.lins@bit3.de}{tristan.lins@bit3.de}

Softwareentwickler seit über 13 Jahren, bei Contao seit über 5 Jahren, Entwickler von Avisota\footnote{\hreflink{http://avisota.org}}, Mitgründer der CCA\footref{fn:cca} und Mitglied der Contao Arbeitsgruppe \uquot{Core-Entwicklung}\footref{fn:contao-workgroup-core}.

% --------------------------------------------------------------------------------
%
%     Subunternehmer
%
% --------------------------------------------------------------------------------

\section{Subunternehmer}
\label{sec:contractor}

Es werden keine Subunternehmer beauftragt.

\newpage

% --------------------------------------------------------------------------------
%
%     Unterstützer
%
% --------------------------------------------------------------------------------

\section{Unterstützer}
\label{sec:backers}

\subsection*{Carolina \uquot{Lucina} Koehn}

Community Forum Moderatorin \\
E-Mail: \href{mailto:ck@kikmedia.de}{ck@kikmedia.de}

\begin{emquote}{}
Der neue \packageManager{} ist der nächste notwendige Schritt für Contao, um komplexe Szenarien abbilden zu können:
Ein funktionierendes, modernes  Paketmanagement, um weitreichende Softwareabhängigkeiten sicher zu warten, Erweiterungen einfach zu installieren, Contao für externe Bibliotheken zu öffnen und damit in der Liga moderner Software mitzuspielen. Das macht Contao nachhaltig und zukunftsfähig.
\end{emquote}

\subsection*{Kim \uquot{k-webdesign} Wormer}

Mitglied in der Contao Association und Contao Community Alliance \\
E-Mail: \href{mailto:info@kim-wormer.de}{info@kim-wormer.de}

\begin{emquote}{}
Mit dem neuen \packageManager{} hat man einen weitaus besseren Überblick über die Abhängigkeiten der einzelnen Erweiterungen.
\end{emquote}

\subsection*{Kirsten \uquot{katgirl} Roschanski}

Mitglied in der Contao Association, Entwicklerin von den Erweiterungen \uquot{avatar} und \uquot{helpdesk}.\\
E-Mail: \href{mailto:kirsten@kirsten-roschanski.de}{kirsten@kirsten-roschanski.de}

\begin{emquote}{}
Mit dem neuen \packageManager{} kann man einfacher neue Versionen - von Entwicklern - beziehen, ohne sich die ganze Zeit auf github zu tummeln. Zudem kommt es durch die strikte Versionsauflösung nicht länger zu Versionkonflikten.
\end{emquote}

\pagebreak

\subsection*{Leo \uquot{leo} Feyer}

Entwickler von Contao.\\
E-Mail: \href{mailto:leo@contao.org}{leo@contao.org}

\begin{emquote}{}
Das ER2 hat gravierende Mängel bei der Abhängigkeitsverwaltung und muss dringend zeitnah ersetzt werden. Eine Composer-basierte Erweiterungsverwaltung ist derzeit die einzige Alternative zum ER2, die mit vertretbarem Aufwand produktionsreif werden kann. Die Integration von Standards wie GitHub, Packagist und Composer ist außerdem der beste Ansatz, so dass ich nicht glaube, dass sich noch weitere Alternativen etablieren werden.
\end{emquote}

\subsection*{Marc \uquot{MacKP} Reimann}

Mitglied der Contao Community Alliance und Forums-Yoda.\\
E-Mail: \href{mailto:reimann@mediendepot-ruhr.de}{reimann@mediendepot-ruhr.de}

\begin{emquote}{}
Der \packageManager{} macht es mir wesentlich einfacher komplexe Erweiterungen zu installieren und aktuell zu halten. Auf lange Sicht ist das genau die richtige Entwicklung: Weg von Closed Source (ER 2) zu einem Open Source Projekt, was aktiv von vielen weiter entwickelt werden kann.
\end{emquote}

\subsection*{Yanick \uquot{Toflar} Witschi}

Mitglied der Contao Association und Mitglied der Contao Arbeitsgruppe \uquot{Core-Entwicklung}\footref{fn:contao-workgroup-core}.\\
E-Mail: \href{mailto:yanick.witschi@terminal42.ch}{yanick.witschi@terminal42.ch}

\begin{emquote}{}
Der neue \packageManager{} ist für Contao 4 eine unverzichtbare Voraussetzung!
\end{emquote}

\pagebreak

\subsection*{Thomas \uquot{planepix} Weitzel}

Mitglied in der Contao Association.\\
Author von \uquot{Mit Contao Webseiten erfolgreich gestalten}\footnote{\hreflink{http://www.think-contao.de/}} \\
und \uquot{Contao für Webdesigner}\footnote{\hreflink{http://www.contao-fuer-webdesigner.de}} \\
E-Mail: \href{mailto:contao@weitzeldesign.de}{contao@weitzeldesign.de}

\begin{emquote}{}
Die zeitnahe Fertigstellung des \packageManager{}s und damit der neuen integrierten Erweiterungsverwaltung ist die konsequente Fortführung der ER-Entwicklung und wird künftig wieder das ER als zentrale Anlaufstelle für alle Erweiterungen für Contao in den Mittelpunkt bringen. Einfachere Bedienbarkeit bei der Pflege von Erweiterungen sowie bessere Übersichtlichkeit für Anwender sind ein klares Ziel.
\end{emquote}

\subsection*{Tim \uquot{timbec} Becker}

Mitglied in der Contao Association und Contao Community Alliance \\
E-Mail: \href{mailto:tim@westwerk.ac}{tim@westwerk.ac}

\begin{emquote}{}
Das \contaoPackageManagerProject{} ist die logische Weiterentwicklung des beliebten ER2 - um die aktuell vorhandenen Defizite auszumerzen und die vorhandene Implementierung (der \composerClient{}) auf noch stabilere Beine zu stellen unterstütze ich die forcierte Weiterentwicklung natürlich.
\end{emquote}

\newpage

% --------------------------------------------------------------------------------
%
%     Projektteam
%
% --------------------------------------------------------------------------------

\section{Projektteam}
\label{sec:team}

Das Projekt wird nicht nur durch die Antragsteller, sondern durch das Team der Contao Community Aliance betreut. Im folgenden sind die Tätigkeitsschwerpunkte der Mitglieder aufgelistet.

\begin{tabularx}{\textwidth}{p{.2\textwidth}X}
\multicolumn{2}{l}{\textbf{Projektleitung}} \\
\hline
\texttt{Tim Becker}          & Organisation und Abstimmung im Projekt. \\
\end{tabularx}

\begin{tabularx}{\textwidth}{p{.2\textwidth}X}
\multicolumn{2}{l}{\textbf{Entwicklung}} \\
\hline
\texttt{Christian Schiffler} & Entwicklung des Grundsystems. \newline
                               Integration in Contao. \newline
                               Administration und Betreuung \store{}. \\
\texttt{Tristan Lins}        & Entwicklung des Grundsystems. \newline
                               Entwicklung der Oberfläche. \newline
                               Schreiben der \documentation{}.
\end{tabularx}

\begin{tabularx}{\textwidth}{p{.2\textwidth}X}
\multicolumn{2}{l}{\textbf{Kundenbetreuung}} \\
\hline
\texttt{Marc Reimann}        & Betreuung im deutschen Community Forum. \\
\texttt{Nicky Hoff}          & Vorträge und Lehrveranstaltungen an der Konferenz und dem Camp. \newline
                               Lektorat der \documentation{}. \\
\end{tabularx}

\begin{figure}[h]
\begin{center}
\href{http://c-c-a.org}{\includegraphics[width=.6\textwidth]{\logo}}
\end{center}
\begin{emquotation}
Die Contao Community Alliance\footnotemark versteht sich als Vereinigung von freien Webschaffenden mit dem Fokus auf die Entwicklung nachhaltiger Lösungen für Contao.
\end{emquotation}
\end{figure}

\footnotetext{{\hreflink{https://c-c-a.org}}}

\newpage

% --------------------------------------------------------------------------------
%
%     Beschreibung
%
% --------------------------------------------------------------------------------

\section{Beschreibung}
\label{sec:description}

\subsection{Zielsetzung}

Aktuell ist die Bedienbarkeit des CCC vor allem technisch orientiert, das liegt vor allem daran, dass die jetzige Version \uquot{von Entwicklern für Entwickler} entwickelt wurde. Ziel ist es die Bedienbarkeit dahingehend deutlich zu verbessern, dass technisch weniger versierte Benutzer keine Probleme mehr haben sollten, den CCC zu bedienen (Usability).

Der CCC ist momentan vom Contao Backend abhängig, was vor allem bei einem fehlgeschlagenen Update oft zu Problemen führt, weil man nicht mehr in der Lage ist die Paketverwaltung zu öffnen. Es soll daher die bisherig geleistete Arbeit heran gezogen werden und anstelle der Backend Integration einen eigenständigen Modus geben, der ohne das Backend funktioniert und mit dem man ein beschädigtes System wieder reparieren kann. Dieser \textit{Standalone Modus} wird - ähnlich wie das Contao-Live-Update\footnote{\hreflink{https://contao.org/de/manual/3.3/installation.html\#live-update-service}} - über eine PHAR Datei realisiert.

Das CCP beinhaltet aktuell Funktionalitäten, die dort einfach nicht rein gehören. Dazu gehören unter anderem die Migrationsroutinen bei Updates. Diese Funktionen sollen in den CCC einfließen und der CCP soll damit so verschlankt werden, dass er auch mit zukünftige Contao Versionen kompatibel ist und bleibt. Das betrifft bspw. die anstehende Contao 4 Version, in der sich die Strukturen ändern und dort einige der vorhandenen Migrationsroutinen im CCP aktuell noch zu Fehlern führen.

Das CCR in Form einer eigenständigen Packagist Installation soll aufgesetzt und für Contao gebrandet werden. Damit wollen wir verhindern, dass Entwickler die jetzt auf Composer umsteigen, nicht noch einen zweiten Umstieg von Packagist.org auf den neuen Contao Paketserver durchführen müssen. In der ersten Form wird der CCR lediglich ein Standard Packagist Server sein. Für 2015 ist ein Förderantrag zur Erweiterung dieses Paketservers geplant, um ihn an die Anforderungen im Contao Umfeld anzupassen.

Die CCD soll sowohl für Entwickler, als auch Anwender und Anbieter kommerzieller Pakete ausgearbeitet werden. Momentan ist die Dokumentation auch noch über viele Quellen verteilt (CCA Website, Contao Wiki, Github Wiki) und soll zentralisiert werden. Die Dokumentation wird in reStructuredText/Sphinx\footnote{\hreflink{http://sphinx-doc.org/}} geschrieben und über Read the Docs\footnote{\hreflink{https://readthedocs.org/}} publiziert. Die Übersetzung erfolgt dann über Transifex\footnote{\hreflink{https://www.transifex.com/}}.

Mit der CCD wurde bereits begonnen, man kann diese unter contao-composer-project.rtfd.org\footnote{\hreflink{http://contao-composer-project.rtfd.org}} einsehen.

\pagebreak

\subsection{Langfristige Ziele und Aufgaben}
\label{sec:long-term-goals}

Die Entwicklungen umfasst nicht nur die nächsten Schritte. In den kommenden Jahren wollen wir weitere Projekte angehen, um das Ökosystem rund um die Contao Paketverwaltung neu aufzubauen.

\subsubsection{Einführung eines erweiterten Packagist Servers für Contao Erweiterungen (=CCR)}

Packagist\footnote{\hreflink{https://packagist.org}} ist der Paketserver zu Composer. Allerdings sind die Möglichkeiten im Vergleich zum aktuellen ER2\footnote{\hreflink{https://contao.org/de/extension-list.html}} doch sehr eingeschränkt. Bspw. die Suche und Filtermöglichkeiten sind momentan noch nicht sonderlich ausgereift. Im Rahmen des CCR möchten wir einen eigenen Paketserver auf Basis von Packagist aufsetzen, diesen entsprechend Branden und auf unsere Bedürfnisse anpassen.\\
Dabei werden wir auch darauf achten, dass wir uns nicht zu sehr von der Originalsoftware entfernen und werden einige Änderungen auch an das Packagist Projekt zurück geben. Daraus erhoffen wir uns einen stärkeren Synergieeffekt. Änderungen die speziell an das Contao Umfeld zugeschnitten werden, verbleiben im CCR und werden nicht an das Packagist Projekt zurück gegeben.

\subsubsection{Erweitern des CCR in einen \uquot{App Store}}

Mit dem aktuellen ER2 ist es möglich kommerzielle Erweiterung zu verbreiten. Allerdings fehlt es an entsprechenden Vermarktungsmöglichkeiten. Diese möchten wir - so ähnlich wie der Contao Theme Store - zukünftig ermöglichen.

\subsubsection{Langfristige Finanzierung}

\paragraph{Serverbetrieb und Administration des CCR} ~\\
Das CCR wird auf dem Community Server der Association betrieben. Falls es auf Dauer notwendig werden sollte, wird die Association einen separaten Server zur Verfügung stellen. Dies ist bereits mit dem Association Vorstand kommuniziert worden.
Die Pflege des Servers wird primär von den zuständigen Serveradministratoren durchgeführt. Die spezifische Administration der Software wird unentgeltlich von Tristan Lins und Christian Schiffler übernommen, solange sich der Aufwand in einem wirtschaftlichen akzeptablen Rahmen bewegt.

\paragraph{Weiterentwicklung und Pflege des CCC+CCP} ~\\
Für die nächsten Jahre soll die Weiterentwicklung und Pflege primär durch die Contao Association im Rahmen dieses Antrags gefördert werden. Wie die finanzielle Absicherung danach aussieht steht noch nicht genau fest.\\
Bereits bei der Association Mitgliederversammlung 2014 waren im Haushalt Gelder für das Composer Projekt vorgesehen, die dann aber dem Topf zur Mittelverwendung zugeführt wurden. Dass nach Umsetzung dieses Antrags möglicherweise wieder Gelder im Haushalt der Association eingeplant und bereitgestellt werden wäre eine denkbare Lösung.

\subsection{Begründung: \uquot{Wie wird Contao durch das Projekt gefördert}}

Contao und die Contao Community profitieren, da Entwickler schneller Updates bereitstellen können und die Updates sich sicherer und schneller installieren lassen. Die Installation inkompatibler Erweiterungen ist damit nicht mehr möglich und somit sinkt die Rate an \uquot{zerstörten} Contao Installation drastisch.

Bereits seit über 2 Jahren wird darüber nachgedacht, das in die Jahre gekommene ER2 zu ersetzen. Durch Composer haben wir nun die Möglichkeit das ER3 endlich realisieren, wodurch Entwicklern und damit auch Contao neue Möglichkeiten zur Verfügung stehen.

\subsection{Lizenzen}

Der CCC und das CCP stehen unter der LGPL-3.0 Lizenz.\\
Das CCR steht unter der MIT Lizenz. \\
Die CCD steht unter der CC-BY-NC-SA 3.0 Unported Lizenz.

\newpage

% --------------------------------------------------------------------------------
%
%     Angebot
%
% --------------------------------------------------------------------------------

\section{Angebot}
\label{sec:offer}

\begin{tabular*}{\textwidth}{@{\extracolsep{\fill} }p{.8\textwidth}r}
\textbf{Position} & \textbf{Preis} \\
\hline

Implementierung des \contaoPackageManager{} für Contao 3.5 LTS und Contao 4.0. \newline
Schreiben der Dokumentation zum \contaoPackageManagerProject{} für \newline
\tabitem Anwender, \newline
\tabitem Kommerzielle Anbieter und \newline
\tabitem Entwickler \newline
Aufsetzen des \store{} und Branding für Contao

& \textbf{10.000 \euro}
\end{tabular*}

% --------------------------------------------------------------------------------
%
%     Umsetzungszeitraum
%
% --------------------------------------------------------------------------------

\section{Umsetzungszeitraum}
\label{sec:timeline}

Bei Beauftragung am 15. Dezember, wird die vollständige Umsetzung bis spätestens zum Erscheinen von Contao 4.0 erfolgen, oder zur Contao Konferenz abhängig davon welches Ereignis zuerst eintritt.

\newpage

% --------------------------------------------------------------------------------
%
%     Ausblick
%
% --------------------------------------------------------------------------------

\section{Ausblick}
\label{sec:forecast}

Nach der Umsetzungen möchten wir uns noch folgenden Positionen stärker widmen.

\paragraph{Bedienbarkeit verbessern} ~\\
Abhängig von dem Feedback möchten wir die Bedienbarkeit für den Otto-Normal-Nutzer weiter verbessern und den \packageManager{} dauerhaft benutzerfreundlich gestalten.

\paragraph{Vermarktungsmöglichkeiten schaffen} ~\\
Für Anbieter kommerzieller Erweiterung möchten wir eine Vermarktungsplatform schaffen, ähnlich dem Contao Theme Store.

\paragraph{Weitreichende Kompatibilität} ~\\
Die Kompatibilität mit \uquot{allen} Shared Hostern zu halten ist eine Aufgabe, der man sich langfristig widmen muss.

\end{document}
