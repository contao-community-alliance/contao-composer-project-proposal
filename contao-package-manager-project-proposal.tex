\documentclass[
paper=a4,
draft=false,%     Entwurfsmodus
fontsize=10pt% Größe der Grundschrift
]{scrartcl}

\def\logo{logos/colorlogo_rgb}% Logo bei Normaldruck.
%\def\logo{logos/colorlogo_cmyk}% Logo bei Profidruck.

\usepackage{cca-page} % design nach vorschrift
\usepackage{multirow}

\newcommand{\contaoPackageManagerProject}{\textbf{\textit{Contao Paket Manager Projekt}}}
\newcommand{\contaoPackageManager}{\textbf{\textit{Contao Paket Manager}}}
\newcommand{\packageManager}{\textbf{\textit{Paket Manager}}}
\newcommand{\moduleInstaller}{\textbf{\textit{Module Installer}}}
\newcommand{\store}{\textbf{\textit{Extension Store}}}
\newcommand{\jsonGenerator}{\textbf{\textit{Composer.json Generator}}}
\newcommand{\documentation}{\textbf{\textit{Dokumentation}}}
\newcommand{\databaseUpdateTool}{\textbf{\textit{Database Update Tool}}}
\newcommand{\migrationToolkit}{\textbf{\textit{Migration Toolkit}}}
\newcommand{\composerClient}{\textbf{\textit{Composer Client}}}
\newcommand{\composerPlugin}{\textbf{\textit{Composer Plugin}}}

\begin{document}

\title{%
An die:\\
Contao Association\\
Sonnhalderain 15\\
CH - 3250 Lyss\\
\vspace*{1cm}%
\href{https://bit3.de}{\includegraphics[width=.2\textwidth]{bilder/bit3}}%
\hspace*{1cm}%
\href{https://www.cyberspectrum.de/}{\includegraphics[width=.2\textwidth]{bilder/cyberspectrum}}%
\vspace*{1cm}\\
Antrag auf Förderung des \\
\textit{Contao Paket Manager Projekts} \\
durch die Contao Association\\%
}
\date{24 Oktober 2014}
\author{bit3 UG, CyberSpectrum}
\maketitle

\pagebreak

% --------------------------------------------------------------------------------
%
%     Präambel
%
% --------------------------------------------------------------------------------

\section*{Präambel}
\label{sec:preamble}

Gefördert werden soll das \contaoPackageManagerProject{}, das durch die Contao Community Alliance betreut wird.

\begin{emquotation}
Ursprünglich hieß dieses Projekt \textbf{\textit{Contao Composer Projekt}} und bestand aus den Teilprojekten \composerClient{} (\texttt{contao-community-alliance/composer-client}) und dem \composerPlugin{} (\texttt{contao-community-alliance/composer-plugin}). Wegen der Benennung kam es oft zu Verwechselung mit dem \textit{PHP Composer Projekt}\footnote{\hreflink{https://getcomposer.org}} was zu viel Verwirrung geführt hat. In einer gemeinsamen Diskussion hat man beschlossen das Projekt um zu benennen\footnotemark um hier eine klare Trennung zu schaffen.
\end{emquotation}

\footnotetext{\hreflink{https://github.com/contao-community-alliance/composer-client/issues/235}}

Das Projekt besteht aus mehreren Teilprojekten:

\begin{itemize}
\item ...dem \packageManager{}, \\
dabei handelt es sich um eine eigenständige Applikation um die in Contao installierten Erweiterungen auf einfache Art und Weise zu verwalten. \\
Den meisten Nutzern dürfte der Vorgänger des \packageManager{}, der \composerClient{} als Menüpunkt \uquot{Paketverwaltung} bekannt sein.

\item ...dem \store{}, \\
dies ist ein an das Contao Ökosystem angepasster Paketserver. Die Benennung lässt vermuten, dass es sich dabei um einen Onlineshop handelt. Dies ist auch das langfristige Ziel, daraus eine Vermarktungsplatform für freie und kommerzielle Erweiterungen zu schaffen - ähnlich dem Theme-Store. Diese Funktionalität ist aber nicht Teil dieses Antrags!

\item ...der \documentation{}.
\end{itemize}

\pagebreak

\subsection*{Welches Ziel hat dieser Antrag?}

Bereits Anfang 2014\footnote{\hreflink{https://contao.org/de/news/zwischenmeldung-zum-extension-repository-3.html}} haben wir in einem klärenden Gespräch festgestellt, dass der vorhandene \composerClient{} zwar funktioniert, aber es vor allem am Ökosystem hapert. Das vorhandene Ökosystem rund um das ER2 ist deutlich umfangreicher als das, was wir um den \composerClient{} bisher aufbauen konnten.

Dieser Antrag befasst sich - soweit es aktuell planbar ist - damit, das Ökosystem rund um den \contaoPackageManager{} aufzubauen, welcher aus dem \composerClient{} hervor geht. So dass wir in absehbarer Zeit - 2015/2016 - das alte ER2 auch tatsächlich ablösen können.

Vor allem die \uquot{mangelnde} Usability, die in den Kommentaren des Newsbeitrags vom 20. März 2014 so oft angesprochen wurde möchten wir \uquot{aus dem Weg schaffen}.

\pagebreak
\tableofcontents
\pagebreak

% --------------------------------------------------------------------------------
%
%     Composer, was ist das?
%
% --------------------------------------------------------------------------------

\section{Composer, was ist das?}
\label{sec:why-composer}

\subsection{Verwirrende Nomenklatur - Was ist Composer?}

\uquot{Composer} hört man als Begriff in der Contao Community in letzter Zeit sehr häufig. Dabei stammt Composer gar nicht aus dem Contao Umfeld. Composer\footnote{\hreflink{https://getcomposer.org}} ist eine sogenannte Paketverwaltung speziell für PHP - nicht nur speziell für Contao. Das Composer Projekt wurde Anfang 2011 von Nils Adermann und Jordi Boggiano initiiert und erfreut sich seit 2012 wachsender Beliebtheit und Verbreitung. Abgesehen von PEAR\footnote{\hreflink{http://pear.php.net/}}, welches nativ auf dem System ausgeführt wird, gab es für PHP über die letzten Jahre hinweg keine vernünftige Paketverwaltung. Composer ist vollständig in PHP geschrieben und lässt sich daher im Gegensatz zu PEAR auch auf restriktiven Systemen (Shared Hoster) verwenden. Zum Composer-Ökosystem gehört unter anderem Packagist\footnote{\hreflink{https://packagist.org/}}, das als primäres Repository für Pakete dient.

\begin{info}
In diesem Dokument ist mit \textbf{Composer} immer das Composer Projekt von Nils und Jordi gemeint.
In der Contao Community ist damit umgangssprachlich der \textbf{Contao Composer Client} gemeint.
Leider führt das des öfteren immer wieder zu Verwirrung. \\
\\
Unter dem Begriff \textbf{Contao Composer Projekt} - umgangssprachlich auch als \textbf{Composer Projekt} abgekürzt - vereinen wir alle zugehörigen Teilprojekte (siehe \nameref{sec:preamble}).
\end{info}

Deshalb wurde beschlossen, das Projekt allgemein in \contaoPackageManagerProject{} um zu benennen. Unter der Haube setzen wir immer noch Composer sein. Auch wenn Composer nicht ganz unstrittig ist, ist es doch Alternativlos und für uns aktuell die beste Möglichkeit schnell zum Ziel zu kommen.

\begin{emquote}{Christian Schiffler, Tristan Lins}
Als wir uns mit der Thematik Thematik \uquot{Paketverwaltung und Abhängigkeitsmanagement} beschäftigten, haben wir festgestellt dass die ganze technisch deutlich komplexer ist, als es auf den ersten Blick den Eindruck macht. Nils und Jordi haben hier eine großartige Vorarbeit geleistet, viele Stunden investiert und obendrein eine große Anzahl an Mitstreitern um sich geschart die regelmäßig dazu beitragen, Composer zu verbessern - wir gehören auch dazu. Composer mag durchaus auch seine Schwächen haben - bspw. relativ hohe Systemanforderungen - aber wir könnten es auch nicht besser machen, besonders nicht bei gleicher Funktionalität.
\end{emquote}

Schon heute wird der \composerClient{} in immer mehr Installationen eingesetzt (>15.000 Downloads\footnote{\hreflink{https://c-c-a.org/composer-counter}}). Einige Erweiterungen die bereits über Composer vertrieben werden, werden im ER2 nur noch sporadisch aktualisiert. Andere Erweiterungen sind gar nicht mehr im ER2 verfügbar. Die bekanntesten Vertreter sind vermutlich Avisota und MetaModels, aber auch einige andere Erweiterungen die nicht aus der Hand der Antragsteller stammen, sind nur noch über Composer verfügbar! Bspw. die Contao Bootstrap Erweiterung\footnote{\hreflink{https://packagist.org/packages/contao-bootstrap/}} von David Molineus.

\newpage

% --------------------------------------------------------------------------------
%
%     Vorteile für Nutzer
%
% --------------------------------------------------------------------------------

\section{Vorteile für Nutzer}
\label{sec:pros-for-users}

\begin{minipage}{.02\linewidth}
  \rotatebox{90}{\textbf{Vorteile auf einen Blick}}
\end{minipage}
\begin{minipage}{.98\linewidth}
  \begin{dinglist}{'64}
  \item Abhängigkeitskonflikte zwischen Erweiterungen werden automatisch aufgelöst und nur kompatible Versionen werden installiert - Stichwort: Versionshopping\footnotemark.
  \item Man erhält schneller Updates und Bugfixes, im Idealfall sofort nach dem ein Bug behoben wurde - Stichwort: Semantische Versionierung\footnotemark.
  \item Im \textit{Standalone Modus} kann man eine defekte Installation auch ohne FTP und händische Korrektur wiederherstellen.
  \item Zukunftssicher dank Open Source und einer breiten Entwicklergemeinschaft - 10 aktive und über 100 weitere Entwickler.
  \end{dinglist}
\end{minipage}

\addtocounter{footnote}{-1}
\footnotetext{Unter Versionshopping verstehen wir die Situation, in der das ER2 dauerhaft eine Erweiterung erst upgraden, danach direkt wieder downgraden will und umgekehrt. Versionshopping entsteht, wenn eine Erweiterung von zwei anderen Erweiterungen als Abhängigkeit bezogen wird und dazu auch noch in 2 unterschiedlichen Versionen. Das ER2 springt dann zwischen den 2 angeforderten Versionen der abhängigen Erweiterung stendig hin und her.\vspace*{3mm}\label{fn:versionshopping}}
\stepcounter{footnote}
\footnotetext{\href{http://semver.org/}{SemVer2 (semver.org)} beschreibt eine Art wie man Versionen einer Software festlegt. Der 3. Level einer Version (also der Z Level: x.y.\textbf{Z}) dient dabei zum versionieren von Bugfixes. Im Idealfall hat jeder Bugfix seine eigene Versionsnummer. D.h. verfolgt ein Entwickler dieses Prinzip und hat er einen Bug behoben, kann er ihn auch gleich in einer neuen Version veröffentlichen. Nutzer müssen nicht mehr auf das nächste \uquot{große Release} warten oder sich mühevoll den Bugfix händisch installieren.\vspace*{3mm}\label{fn:semver}}

\textit{Ein schlanker Core und hochwertige/umfangreiche Erweiterungen}, dafür steht Contao seit 2006. Um die Bereitstellung des Erweiterungskatalogs kümmert sich seit geraumer Zeit das so genannte \uquot{Extension Repository 2} - kurz \uquot{ER2}. Das ER2 stellt aktuell die Erweiterungsliste auf Contao\footnote{\hreflink{https://contao.org/de/extension-list.html}}, so wie den Erweiterungskatalog und die Erweiterungsverwaltung im Contao Backend. Doch das ER2 hat seine Schwächen, so werden Abhängigkeiten nicht richtig aufgelöst - Stichwort: Versionshopping\footref{fn:versionshopping} - oder es lassen sich mit der Contao Version oder mit anderen Erweiterungen inkompatible Erweiterungen installieren.

Unterm Strich muss man sagen, dass die Vorteile von Composer primär die Entwickler betreffen. Doch was für den Entwickler ein Vorteil ist, ist letztlich auch für den Nutzer ein Vorteil. Aber auch die Schwächen des aktuellen CCC für die Nutzer sind uns bewusst und gerade diese wollen wir mit Hilfe dieses Antrags beheben. Dazu zählen unter anderem eine deutliche Verbesserung der Übersichtlichkeit beim Finden, Installieren und Verwalten von Erweiterungen. Zum anderen wollen wir die Kompatibilität zu den zahlreichen Shared Hostern erhöhen.

Aber worum geht es konkret?

Durch die veraltete und unzureichende Implementierung der Abhängigkeitsverwaltung im ER2 ist es aktuell möglich, wenn nicht gar wahrscheinlich, dass sich Nutzer Erweiterungen die zu einander oder zur aktuellen Contao Version inkompatibel sind installieren und somit potentiell die Installation lahm legen - Stichwort: \uquot{gelbe Box}. Diese Probleme werden durch Composer weitestgehend vermieden, so lange die Entwickler selbst keinen Mist bauen. Denn Composer hat eine \uquot{strikte} und zugleich \uquot{offener} Auflösung der Abhängigkeiten. Richtig angewendet sorgt das dafür, dass man als Nutzer immer mit dem neuesten, stabilsten und zu meiner Installation kompatiblen Set an Erweiterungen arbeiten kann.

Sollte es dennoch zu Problemen kommen - den auch Composer kann uns nicht vor allen Problemen schützen - und sollte das Backend dann nicht mehr nutzbar sein, kann man mit Hilfe des \textit{Standalone Modus} die defekte Erweiterung auch wieder entfernen und das Backend \uquot{wiederbeleben}. Den \textit{Standalone Modus} kann man sich ähnlich wie das Install Tool vorstellen, nur halt für die Erweiterungsverwaltung. Mit dem ER2 ist dies aktuell nicht möglich und man muss von Hand via FTP nachhelfen. Gerade für nicht-Entwickler ist hier nicht selten Nachfragen im Forum, IRC, Twitter und Co. erforderlich.

Ein weiterer Punkt ist die Veröffentlichung von neuen Versionen. Für Entwickler ist dies aktuell relativ zeitaufwendig, abgesehen vom \uquot{taggen}\footnote{Damit bezeichnet man das Markieren einer Version zu einem gewissen Stand des Quellcodes.} der Version müssen Entwickler die neue Version auch noch auf der Contao Website eintragen, von Hand hochladen\footnote{Es gibt zwar mitlerweile einen github Import, der aber nicht in jeder Situation funktioniert.} und dann auch noch veröffentlichen. Composer und Packagist\footnote{\href{https://packagist.org}{Packagist.org} ist der offizielle Paketserver zu Composer.} automatisieren den Prozess nach dem \uquot{taggen} vollständig. Das macht es Entwicklern einfacher, gerade Bugfixes schnell und unkompliziert zu veröffentlichen - Stichwort: Semantische Versionierung\footref{fn:semver}.

Zu guter Letzt haben wir mit dem ER2 noch ein großes Problem, was den wenigsten bekannt ist. Das ER2 wurde ursprünglich exklusiv für Contao entwickelt und ist technisch stark veraltet. Keiner der Contao Entwickler mag sich an diesen \uquot{Dinosaurier} dran begeben, was dazu führt dass die Weiterentwicklung des ER2 gänzlich zum erliegen gekommen ist. Mit Composer setzen wir auf Open Source Software, die technisch auf dem neuesten Stand ist und die wir an unsere Bedürfnisse anpassen können, wenn es notwendig wird.

\newpage

% --------------------------------------------------------------------------------
%
%     Vorteile für Entwickler
%
% --------------------------------------------------------------------------------

\section{Vorteile für Entwickler}
\label{sec:pros-for-developers}

\begin{minipage}{.02\linewidth}
  \rotatebox{90}{\textbf{Vorteile auf einen Blick}}
\end{minipage}
\begin{minipage}{.98\linewidth}
  \begin{dinglist}{'64}
  \item Abhängigkeitskonflikte zwischen Erweiterungen werden zuverlässig erkannt und vermieden.
  \item Große Auswahl an Contao Erweiterungen und PHP Bibliotheken die man als Abhängigkeit mit installieren lassen kann (mehr als 40.000) - Stichwort: Synergie.
  \item Die große Community sorgt für eine ständige Weiterentwicklung von Composer und dessen Ökosystems - Stichwort: Zukunftssicher.
  \item Einfache und \uquot{automatisierte} Veröffentlichung von neuen Versionen - Stichwort: Semantische Versionierung.
  \end{dinglist}
\end{minipage}

Den meisten Entwicklern sollten die Schwächen des ER2 bekannt sein. Die schlechte und unzureichende Auflösung von Abhängigkeiten zwischen Erweiterungen sorgt regelmäßig für Probleme. Und die aufwendige Pflege zur Veröffentlichung neuer Versionen ist nichts weiter als eine große ABM\footnote{Arbeitsbeschaffungsmaßnahme}.

Composer beherrscht das Auflösen von Abhängigkeiten perfekt. Mit Hilfe der \textit{Package links}\footnote{\hreflink{https://getcomposer.org/doc/04-schema.md\#package-links}} kann man als Entwickler die Abhängigkeiten genau zu definieren. Hinzu kommt dass man nicht nur die Abhängigkeiten auf andere Contao Erweiterungen definieren kann. Es ist sogar möglich mehr als 40.000 PHP Bibliotheken zu nutzen. Daraus ergeben sich für Entwickler ganz neue Möglichkeiten - Stichwort: Synergie.

Die große Community sorgt außerdem dafür, dass Composer und sein Ökosystem ständig weiterentwickelt werden. Bei Problemen hilft also nicht nur die Contao Community, sondern auch die Composer Community. Im \#composer\footnote{\hreflink{irc://irc.freenode.net/\#composer}} IRC Channel tummeln sich durchschnittlich 100 bis 150 Nutzer täglich, im \#composer-dev\footnote{\hreflink{irc://irc.freenode.net/\#composer-dev}} Channel sind es um die 30 bis 50 Entwickler täglich. Zum Vergleich bei Contao tummeln sich in \#contao\footnote{\hreflink{irc://irc.freenode.net/\#contao}} nur um die 30 bis 50 Nutzer und in \#contao.dev\footnote{\hreflink{irc://irc.freenode.net/\#contao.dev}} nur 15 bis 25 Entwickler. Die große, wachsende Composer Community liefert auch immer wieder neue \uquot{Best-Practice} Beispiele, Anleitungen und Tutorials.

Veröffentlichen von neuen Versionen wird mittels Composer so einfach wie nie zuvor, denn die Erweiterungen können dort belassen werden, wo sie heutzutage eh meist schon sind: Auf github, bitbucket oder einem anderen Anbieter. Einzige Bedingung dabei ist, dass der Quellcode öffentlich erreichbar ist. Im Gegensatz zum ER2, wo jede Version einzeln von Hand auf dem ER2 Server gepflegt werden muss, reicht bei Composer ein einmaliges Registrieren auf dem Paketserver (bspw. packagist.org). Neue Versionen werden ausschließlich durch setzen von Tags\footnote{Markierung eines exakten Standes der Entwicklung unter einem Namen, oftmals eine Versionsnummer in der Art von 1.0.0} im VCS System erzeugt. Da für gewöhnlich diese Tags von Entwicklern ohnehin gesetzt werden, fällt die ganze Verwaltung der Versionen auf dem Paketserver weg. Man kann also mit Recht behaupten: Es fällt Arbeit weg und kommt keine neue Arbeit hinzu, um die einzelnen Versionen zu verwalten!

\begin{info}
* Für Anbieter kommerzieller Erweiterungen gibt es die Möglichkeit - über private Satis Repositories und Artifact Pakete - den eigenen Code vor fremden Zugriffen zu schützen und trotzdem die Erweiterung über Composer zu verteilen. Diese Möglichkeiten werden im Rahmen der Dokumentation genau erklärt. Außerdem soll es langfristig im \store{} die Möglichkeit geben, kostenpflichtige Erweiterung zu vermarkten und zu verteilen.
\end{info}

\newpage

% --------------------------------------------------------------------------------
%
%     Antragsteller
%
% --------------------------------------------------------------------------------

\section{Antragsteller}
\label{sec:proposer}

\subsection*{Christian \uquot{xtra} Schiffler}

\begin{wrapfigure}{r}{.3\textwidth}
  \vspace{-50pt}
  \hfill
  \includegraphics[width=.2\textwidth]{bilder/cyberspectrum}
\end{wrapfigure}

Unternehmen: CyberSpectrum (\href{https://www.cyberspectrum.de}{www.cyberspectrum.de}) \\
E-Mail: \href{mailto:c.schiffler@cyberspectrum.de}{c.schiffler@cyberspectrum.de}

Softwareentwickler seit über 21 Jahren, bei Contao seit über 6 Jahren, Entwickler von MetaModels\footnote{\hreflink{https://now.metamodel.me}}, Betreuer des Contao Community Wiki\footnote{\hreflink{http://www.contaowiki.org/}}, Mitgründer der CCA\footnote{\hreflink{https://c-c-a.org/}\label{fn:cca}} und Mitglied der Contao Arbeitsgruppe \uquot{Core-Entwicklung}\footnote{\hreflink{https://contao.org/de/team.html\#workgroup-core}\label{fn:contao-workgroup-core}}.

\subsection*{Tristan \uquot{tril} Lins}

\begin{wrapfigure}{r}{.3\textwidth}
  \vspace{-40pt}
  \hfill
  \includegraphics[width=.2\textwidth]{bilder/bit3}
\end{wrapfigure}

Unternehmen: bit3 UG (\href{https://bit3.de}{bit3.de}) \\
E-Mail: \href{mailto:tristan.lins@bit3.de}{tristan.lins@bit3.de}

Softwareentwickler seit über 13 Jahren, bei Contao seit über 5 Jahren, Entwickler von Avisota\footnote{\hreflink{http://avisota.org}}, Mitgründer der CCA\footref{fn:cca} und Mitglied der Contao Arbeitsgruppe \uquot{Core-Entwicklung}\footref{fn:contao-workgroup-core}.

% --------------------------------------------------------------------------------
%
%     Subunternehmer
%
% --------------------------------------------------------------------------------

\section{Subunternehmer}
\label{sec:contractor}

Es werden keine Subunternehmer beauftragt.

\newpage

% --------------------------------------------------------------------------------
%
%     Unterstützer
%
% --------------------------------------------------------------------------------

\section{Unterstützer}
\label{sec:backers}

\subsection*{Carolina \uquot{Lucina} Koehn}

Community Forum Moderatorin \\
E-Mail: \href{mailto:ck@kikmedia.de}{ck@kikmedia.de}

\begin{emquote}{}
Der neue \packageManager{} ist der nächste notwendige Schritt für Contao, um komplexe Szenarien abbilden zu können:
Ein funktionierendes, modernes  Paketmanagement, um weitreichende Softwareabhängigkeiten sicher zu warten, Erweiterungen einfach zu installieren, Contao für externe Bibliotheken zu öffnen und damit in der Liga moderner Software mitzuspielen. Das macht Contao nachhaltig und zukunftsfähig.
\end{emquote}

\subsection*{Kim \uquot{k-webdesign} Wormer}

Mitglied in der Contao Association und Contao Community Alliance \\
E-Mail: \href{mailto:info@kim-wormer.de}{info@kim-wormer.de}

\begin{emquote}{}
Mit dem neuen \packageManager{} hat man einen weitaus besseren Überblick über die Abhängigkeiten der einzelnen Erweiterungen.
\end{emquote}

\subsection*{Kirsten \uquot{katgirl} Roschanski}

Mitglied in der Contao Association, Entwicklerin von den Erweiterungen \uquot{avatar} und \uquot{helpdesk}.\\
E-Mail: \href{mailto:kirsten@kirsten-roschanski.de}{kirsten@kirsten-roschanski.de}

\begin{emquote}{}
Mit dem neuen \packageManager{} kann man einfacher neue Versionen - von Entwicklern - beziehen, ohne sich die ganze Zeit auf github zu tummeln. Zudem kommt es durch die strikte Versionsauflösung nicht länger zu Versionkonflikten.
\end{emquote}

\pagebreak

\subsection*{Leo \uquot{leo} Feyer}

Entwickler von Contao.\\
E-Mail: \href{mailto:leo@contao.org}{leo@contao.org}

\begin{emquote}{}
Das ER2 hat gravierende Mängel bei der Abhängigkeitsverwaltung und muss dringend zeitnah ersetzt werden. Eine Composer-basierte Erweiterungsverwaltung ist derzeit die einzige Alternative zum ER2, die mit vertretbarem Aufwand produktionsreif werden kann. Die Integration von Standards wie GitHub, Packagist und Composer ist außerdem der beste Ansatz, so dass ich nicht glaube, dass sich noch weitere Alternativen etablieren werden.
\end{emquote}

\subsection*{Marc \uquot{MacKP} Reimann}

Mitglied der Contao Community Alliance und Forums-Yoda.\\
E-Mail: \href{mailto:reimann@mediendepot-ruhr.de}{reimann@mediendepot-ruhr.de}

\begin{emquote}{}
Der \packageManager{} macht es mir wesentlich einfacher komplexe Erweiterungen zu installieren und aktuell zu halten. Auf lange Sicht ist das genau die richtige Entwicklung: Weg von Closed Source (ER 2) zu einem Open Source Projekt, was aktiv von vielen weiter entwickelt werden kann.
\end{emquote}

\subsection*{Yanick \uquot{Toflar} Witschi}

Mitglied der Contao Association und Mitglied der Contao Arbeitsgruppe \uquot{Core-Entwicklung}\footref{fn:contao-workgroup-core}.\\
E-Mail: \href{mailto:yanick.witschi@terminal42.ch}{yanick.witschi@terminal42.ch}

\begin{emquote}{}
Der neue \packageManager{} ist für Contao 4 eine unverzichtbare Voraussetzung!
\end{emquote}

\pagebreak

\subsection*{Thomas \uquot{planepix} Weitzel}

Mitglied in der Contao Association.\\
Author von \uquot{Mit Contao Webseiten erfolgreich gestalten}\footnote{\hreflink{http://www.think-contao.de/}} \\
und \uquot{Contao für Webdesigner}\footnote{\hreflink{http://www.contao-fuer-webdesigner.de}} \\
E-Mail: \href{mailto:contao@weitzeldesign.de}{contao@weitzeldesign.de}

\begin{emquote}{}
Die zeitnahe Fertigstellung des \packageManager{}s und damit der neuen integrierten Erweiterungsverwaltung ist die konsequente Fortführung der ER-Entwicklung und wird künftig wieder das ER als zentrale Anlaufstelle für alle Erweiterungen für Contao in den Mittelpunkt bringen. Einfachere Bedienbarkeit bei der Pflege von Erweiterungen sowie bessere Übersichtlichkeit für Anwender sind ein klares Ziel.
\end{emquote}

\subsection*{Tim \uquot{timbec} Becker}

Mitglied in der Contao Association und Contao Community Alliance \\
E-Mail: \href{mailto:tim@westwerk.ac}{tim@westwerk.ac}

\begin{emquote}{}
Das \contaoPackageManagerProject{} ist die logische Weiterentwicklung des beliebten ER2 - um die aktuell vorhandenen Defizite auszumerzen und die vorhandene Implementierung auf noch stabilere Beine zu stellen unterstütze ich die forcierte Weiterentwicklung natürlich.
\end{emquote}

\newpage

% --------------------------------------------------------------------------------
%
%     Projektteam
%
% --------------------------------------------------------------------------------

\section{Projektteam}
\label{sec:team}

Das Projekt wird nicht nur durch die Antragsteller, sondern durch das Team der Contao Community Aliance betreut. Im folgenden sind die Tätigkeitsschwerpunkte der Mitglieder aufgelistet.

\begin{tabularx}{\textwidth}{p{.2\textwidth}X}
\multicolumn{2}{l}{\textbf{Projektleitung}} \\
\hline
Tim Becker          & Organisation und Abstimmung im Projekt. \\
\end{tabularx}

\begin{tabularx}{\textwidth}{p{.2\textwidth}X}
\multicolumn{2}{l}{\textbf{Entwicklung}} \\
\hline
Christian Schiffler & Entwicklung des \packageManager{}, \newline
                      Schwerpunkt: Integration in Contao. \newline
                      Administration und Betreuung \store{}. \\
Tristan Lins        & Entwicklung des \packageManager{}, \newline
                      Schwerpunkt: Oberflächendesign und Bedienbarkeit. \newline
                      Schreiben der \documentation{}.
\end{tabularx}

\begin{tabularx}{\textwidth}{p{.2\textwidth}X}
\multicolumn{2}{l}{\textbf{Kundenbetreuung}} \\
\hline
Marc Reimann        & Betreuung im deutschen Community Forum. \\
Nicky Hoff          & Vorträge und Lehrveranstaltungen an der Konferenz und dem Camp. \newline
                      Lektorat der \documentation{}. \\
\end{tabularx}

\begin{figure}[h]
\begin{center}
\href{http://c-c-a.org}{\includegraphics[width=.6\textwidth]{\logo}}
\end{center}
\begin{emquotation}
Die Contao Community Alliance\footnotemark versteht sich als Vereinigung von freien Webschaffenden mit dem Fokus auf die Entwicklung nachhaltiger Lösungen für Contao.
\end{emquotation}
\end{figure}

\footnotetext{{\hreflink{https://c-c-a.org}}}

\newpage

% --------------------------------------------------------------------------------
%
%     Beschreibung
%
% --------------------------------------------------------------------------------

\section{Beschreibung}
\label{sec:description}

\subsection{Zielsetzung}

Der \packageManager{} soll aus dem bestehenden \composerClient{} entstehen.

Der \packageManager{} soll das zentrale Paketverwaltungstool für Contao 3.5 LTS und Contao 4.0 und folgende Versionen werden.

\begin{danger}
\textbf{Hinweis} Das ER2 - im Backend unter den Menüpunkten \uquot{Erweiterungskatalog} und \uquot{Erweiterungsverwaltung} bekannt - kann mit Contao 4 nicht mehr eingesetzt werden!
\end{danger}

Der \store{} soll die zentrale Stelle Platform zum vermarkten, bewerben und verbreiten von Contao Erweiterungen werden.

\subsection{Begründung: \uquot{Wie wird Contao durch das Projekt gefördert}}

Das aktuelle ER2 ist mit Contao 4.0 aufgrund der technischen Umstellungen nicht mehr Einsatzfähig. Contao 4.0 nutzt selbst Composer und der \packageManager{} wird die einzige Möglichkeit sein, Pakete und Erweiterungen in Contao 4.0 bequem über eine Oberfläche zu verwalten.

\pagebreak

\subsection{Lizenzen}

Der \packageManager{} stehen unter der LGPL-3.0 Lizenz.\\
Das \store{} steht unter der MIT Lizenz. \\
Die \documentation{} steht unter der CC-BY-NC-SA 3.0 Unported Lizenz.

\subsection{Systemvoraussetzungen}

Die Systemvoraussetzungen um den \packageManager{} einsetzen zu können.

\begin{dinglist}{'64}
\item PHP 5.4 oder neuer
\item PHAR Support
\item Contao 3.5 LTS oder Contao 4.0 oder neuer
\end{dinglist}

\newpage

% --------------------------------------------------------------------------------
%
%     Angebot
%
% --------------------------------------------------------------------------------

\section{Angebot}
\label{sec:offer}

\begin{dinglist}{'64}
\item Implementierung des \contaoPackageManager{} für Contao 3.5 LTS und Contao 4.0.
\item Schreiben der Dokumentation zum \contaoPackageManagerProject{} für Anwender, Kommerzielle Anbieter und Entwickler
\item Aufsetzen des \store{} und Branding für Contao.
\end{dinglist}

\hfill Summe: \textbf{10.000 \euro}

% --------------------------------------------------------------------------------
%
%     Umsetzungszeitraum
%
% --------------------------------------------------------------------------------

\section{Umsetzungszeitraum}
\label{sec:timeline}

Bei Beauftragung am 15. Dezember, wird die vollständige Umsetzung bis spätestens zum Erscheinen von Contao 4.0 erfolgen, oder zur Contao Konferenz abhängig davon welches Ereignis zuerst eintritt.

\newpage

% --------------------------------------------------------------------------------
%
%     Ausblick
%
% --------------------------------------------------------------------------------

\section{Ausblick}
\label{sec:forecast}

Nach der Umsetzungen möchten wir uns noch folgenden Positionen stärker widmen.

\paragraph{Bedienbarkeit verbessern} ~\\
Abhängig von dem Feedback möchten wir die Bedienbarkeit für den Otto-Normal-Nutzer weiter verbessern und den \packageManager{} dauerhaft benutzerfreundlich gestalten.

\paragraph{Vermarktungsmöglichkeiten schaffen} ~\\
Für Anbieter kommerzieller Erweiterung möchten wir eine Vermarktungsplatform schaffen, ähnlich dem Contao Theme Store.

\paragraph{Weitreichende Kompatibilität} ~\\
Die Kompatibilität mit \uquot{allen} Shared Hostern zu halten ist eine Aufgabe, der man sich langfristig widmen muss.

\end{document}
